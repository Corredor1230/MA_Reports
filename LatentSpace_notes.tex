\section{Latent Space Interpolation}

\cite{latentSpaceInterpolation} used as main reference for the research.

\subsection{Important bullet points}

\begin{itemize}
    \item Interpolation between timbres for a synthesizer should be more nuanced than a simple crossfade.
    \item This work was carried out with a Variational Auto-Encoder (VAE) dedicated to preset interpolation.
    \item This is helpful as it establishes an initial framework for sound design.
    \item Presets are handled as sequences of parameters using \textit{multi-head attention networks}.
    \item Includes an objective morphing evaluation method based on audio features.
    \item Mentions harmonic-percussive source separation from the following article: \cite{Harmonic_Percussive_Separation}
    \item Existing synthesizers and research either underperform under slightly unexpected inputs, or yield lower quality outputs, or cannot be used in real-life contexts using MIDI input.
    \item This research focuses on smooth transitions for synthesis.
    \item In this case synthesizers are treated as black-box de-facto systems, which are not differentiable.
\end{itemize}

\subsection{The proposed model}

\subsubsection{Synthesizer and Dataset}

Presets and synthesized audio:

\begin{itemize}
    \item 30k presets (80\% training, 10\% validation and 10\% for testing). 
    \item Volume, transpose and filter were not altered.
    \item Presets are rendered to 16kHz audio with a single note (MIDI 56) and velocity (75). 
    \item 257-band mel-spectrograms are used during training.
\end{itemize}

Audio features:

\begin{itemize}
    \item Audio Commons Timbral Models (ACTM) and Timbre Toolbox \cite{TimbreToolbox} allow the extraction of 8 and 46 features respectively. Then reduced to 6 and 32 due to high'correlation between some of them.
    \item "These audio features act as a proxy for the human perception of timbre." \cite[p. 4]{latentSpaceInterpolation} 
    \item Name of the model introduced: SPINVAE-2.
\end{itemize}

The structure:

\begin{itemize}
    \item VAE with an extra mel-spectrogram decoder.
\end{itemize}

Attribute-based regularization: Timbre Loss

\begin{itemize}
    \item $L_DKL$ is not enough to guarantee that timbre characteristics are encoded.
    \item "Latent coefficients have been shown not to easily relate to perception"
    \item Models are not directly trained for interpolation. Rather they get interpolated latent spaces and must decode those into audio.
    \item Two regularization methods: 
    \item \begin{itemize}
        \item \cite{Disentanglement} "minimizes the binary cross-entropy between S(a) and S(u), where S denotes the logistic sigmoid."
        \item \cite{Attribute_Based_Regularization} "enforces monotonic relationships between timbre attributes and some latent dimensions."
    \end{itemize} 
\end{itemize}

Training procedure:

\begin{itemize}
    \item A bi-modal (mel-spectrograms and presets) VAE has been pre-trained.
    \item A CNN is added to the model and its oututs are re-shaped and summed to the u and sigma vectors.
    \item SPINVAE-1 \cite{SPINVAE1}
    \item First only CNN encoder and decoder are trained from a mel-spectrogram dataset from NSynth notes, 2.2k Surge synthesizer pathes and 24k Dexed presets.
    \item The weights of embedding, encoder and decoder layers are used as initial weights for all models.
    \item Fine-tuning is performed without CNN encoder.
\end{itemize}

\subsection{Objective analysis}

While the evaluation of linear interpolation is relatively simple, it is more complex to evaluate the interpolation of audio.

\cite{SPINVAE1} talks about smoothness as a measure for interpolation.

According to \cite{Sound_Morphing}, "an interpolation is perceptually linear when timbre feature values change linearly". 

Therefore \textit{both linearity and smoothness must be measured}.



In this case 

\subsection{Newly learned concepts from this document}

\begin{itemize}
    \item Multi-head attention networks:
    \item Regularization term based on timbre attributes
    \item Latent Timbre Synthesis (LTS) \cite{Latent_Timbre} 
    \item Dexed
    \item Zero-mean, unit-variance normalized using statistics of the training set
    \item Attention masks, hidden tokens
    \item Transformer-based VAEs
    \item Linear projection
    \item Discretized Logistic Mixture
    \item Regularized dimensions vs latent dimensions
\end{itemize}

\subsection{How it relates to my research}

