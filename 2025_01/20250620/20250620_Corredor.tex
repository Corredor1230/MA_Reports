\documentclass[a4paper, 12pt]{article}
%\usepackage[utf8]{inputenc}  % Ensures proper encoding for non-Japanese text
\usepackage{xeCJK}           % Allows Japanese characters
\usepackage{geometry}
\usepackage{graphicx}
\usepackage{hyperref}
\usepackage{caption}
\usepackage{subcaption}

\geometry{margin=1in}

\setCJKmainfont{MS Mincho}  
\title{進捗報告}
\author{コレドール・パブロ}
\date{2025年6月20日}

\begin{document}

\maketitle

\section*{近況}
\begin{itemize}
    \item 「東ローマ帝国社会と政府の歴史」をやっと読み終わり、これからはマーシャル・ホジソンによる「イスラームの冒険:世界文明における良心と歴史」を読んでみます。
    \item 夏です・・・(泣)
\end{itemize} 

\section*{やったこと}
\begin{itemize}
    \item インターンシップに応募した(意外と時間かかりすぎて他の事はほぼ出来なかった)。
    \item ギターのデータを増やすためにピッチシフトのデータ拡張や新しいサンプルを収録中です。
    \item マンドリンの収録されているサンプルの前処理を実行し始め、新しいサンプルも収録中です。
\end{itemize}

\section*{考えていること}

トレーニングスペクトルグラムをエキスパンダー、入力音源をハイパスフィルタで処理して、トレーニングしたらどんな出力ができるか気になります。現在使用しているモデルではスペクトルグラムはノイズが高いため、Griffin Limの出力もSNRが低いかもしれない。

% \begin{figure}
% \centering
% \begin{subfigure}{1.0\textwidth}
%   \centering
%   \includegraphics[width=0.9\linewidth]{../media/training_data/20250613/film_epoch320_pitch.png}
%   \caption{現在使用しているモデルの出力}
%   \label{fig:0606modelWave}
% \end{subfigure}
% \vspace{5mm}
% \begin{subfigure}{1.0\textwidth}
%   \centering
%   \includegraphics[width=0.9\linewidth]{../media/training_data/20250613/film_epoch320_spectro.png}
%   \caption{そのスペクトルグラム}
%   \label{fig:0606modelSpectro}
% \end{subfigure}
% \caption{モデル}
% \label{fig:0606Model}
% \end{figure}

% \begin{figure}
% \centering
% \begin{subfigure}{1.0\textwidth}
%   \centering
%   \includegraphics[width=0.9\linewidth]{../media/training_data/20250613/elec_pitch.png}
%   \caption{トレーニングデータとして使用しているギターサンプル}
%   \label{fig:eGuitWave}
% \end{subfigure}%
% \vspace{5mm}
% \begin{subfigure}{1.0\textwidth}
%   \centering
%   \includegraphics[width=0.9\linewidth]{../media/training_data/20250613/elec_spectro.png}
%   \caption{そのスペクトルグラム}
%   \label{fig:eGuitSpectro}
% \end{subfigure}
% \caption{エレキギター}
% \label{fig:elecGuit}
% \end{figure}


\section*{やること}
\begin{itemize}
    \item トレーニング続きます。
    \item 収録続きます。
    \item 夏を生き残ってみます。
\end{itemize}

%Bibliography (duh)
\bibliography{../ref}
\bibliographystyle{apalike}

\end{document}