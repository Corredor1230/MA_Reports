\documentclass[a4paper, 12pt]{article}
%\usepackage[utf8]{inputenc}  % Ensures proper encoding for non-Japanese text
\usepackage{xeCJK}           % Allows Japanese characters
\usepackage{geometry}
\usepackage{graphicx}
\usepackage{hyperref}

\geometry{margin=1in}

\setCJKmainfont{MS Mincho}  % You can change this to a different Japanese font if needed

\title{進捗報告}
\author{コレドール・パブロ}
\date{2024年}

\begin{document}

\maketitle

\section*{近況}
% Write your recent findings here.
\begin{itemize}
    \item 東ローマ帝国の歴史について読めば読むほど、人間の社会とその支配者は過去を繰り返す傾向が強いことに気づいた。
    \item 新しい本棚を買って、本、雑誌等の整理をした。
\end{itemize} 

\section*{やったこと}
\begin{itemize}
    \item \cite{latentSpaceInterpolation}を拝読した。
    \item \cite{TimbreToolbox}を拝読開始した。
    \item データベースの前処理はまだ完了できていない。
\end{itemize}

\section*{考えていること}

前処理はきっと楽で、すぐに終わるだろうと思ったのに・・・

まだ終わっていない状態です。

最初はLibrosaを使用し、ファイルをノーマライズ、編集、名付けようと思っていたが、トランジェント認識はあまり上手くいかなくて、結局自分でしているので、なかなか終わらない。

最初のトレーニングのデータベースをアコギターとエレキギターのフォルテのみにする予定ですから、このペースで後はそんなに長くならないはずです。

\cite{latentSpaceInterpolation}によってMEL-SpectrogramとTimbre Toolbox\cite{TimbreToolbox}の分析方法でモデルのトレーニングとシンセサイザーの生成もできるらしいので、DDSPのモデルの出力によって、その文献の方法も試そうと思っている。

\section*{やること}
% Outline future steps here.
\begin{itemize}
    \item 前処理を完了する
    \item モデルのトレーニングを今度こそ開始する
\end{itemize}

% Add your bibliography here.
\bibliography{../ref}
\bibliographystyle{apalike}

\end{document}