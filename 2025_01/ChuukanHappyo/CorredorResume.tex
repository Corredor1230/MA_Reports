\documentclass[a4paper, 12pt]{article}
%\usepackage[utf8]{inputenc}  % Ensures proper encoding for non-Japanese text
\usepackage{xeCJK}           % Allows Japanese characters
\usepackage{geometry}
\usepackage{graphicx}
\usepackage{hyperref}
\usepackage{caption}
\usepackage{subcaption}

\geometry{margin=1in}

\setCJKmainfont{MS Mincho}  

\title{機械学習により、撥弦楽器の中間音を創り出す\\シンセサイザーの開発}
\author{丸井研究室・M1・Juan Pablo Corredor\\東京芸術大学音楽研究科}
\date{2025年07月18日}

\begin{document}

\maketitle

\section{研究の目的}
\begin{itemize}
  \item 撥弦楽器(ギター、琵琶、マンドリンなど)の音色を補間し、新たな音を創出するシンセサイザーを機械学習により開発。
  \item 音色補間の概念を探求。
  \item 機械学習の分野においての自主録音データベースの可能性を検証。
  \item ミュージシャンのための音楽的知見に基づいたツールの創造。
\end{itemize}

\section{背景と意義}
\begin{itemize}
  \item 現代の音楽制作において、新しい音の創出は重要な表現手段。
  \item 機械学習と統計分析を活用することで、従来にはない音色空間を設計可能。
  \item 音色の補間は、実験音楽や電子音楽において有用なアプローチ。
\end{itemize}

\section{方法論}

\subsection*{研究方法}
\begin{itemize}
  \item 信号処理\cite{DDSP}、VAE\cite{SPINVAE1}や畳み込みニューラルネットワーク\cite{CNNs}など最適な機械学習手法を検討。
  \item 類似研究および既存ソフトウェアのベストプラクティスを調査。
\end{itemize}

\subsection*{開発方法}
\begin{itemize}
  \item 開発環境の構築と動作検証。
  \item 各種モデルの性能評価と改良。
  \item 音楽演奏のようなリアルタイム環境での検証。
  \item ハードウェア拡張の可能性も検討。
\end{itemize}

\section{音色生成プロセス}

\begin{enumerate}
  \item データの準備(録音・前処理)\cite{DDSP_Piano}
  \item 無音カット、ダウンサンプリング、フィルター処理
  \item データの単純化(名称、弦番号、ダイナミクス抽出)
  \item ノイズ、信号、トランジェント\cite{sitrano}の抽出
  \item モデルへの入力と学習
  \item 出力の再構築と誤差調整
  \item 学習の繰り返し
  \item 音声データへの変換
\end{enumerate}

\section{システムの使用例}
\begin{itemize}
  \item 補間係数を選択
  \item MIDIキーボードでノートを演奏
  \item モデルが音色と周波数に応じたデータを生成
  \item 行列を音声に変換して出力
\end{itemize}

\section{結果例と検証}

\begin{itemize}
  \item アコースティックギターとエレキギターの音を学習データとして使用。
  \item モデルの各段階(初期、後期)での出力例を比較。
  \item 音声およびスペクトログラムの違いを可視化。
  \item 今後の改善点:ノイズ、トランジェント、サイン波の分析方法。
\end{itemize}

\section{今後の展望}
\begin{itemize}
  \item より多様な楽器の追加とデータ収集。
  \item 補間アルゴリズムの改良による音色表現の向上。
  \item ライブパフォーマンス向けのリアルタイム応答の強化。
  \item ハードウェア連携による拡張性の模索。
\end{itemize}

\bibliography{../ref}
\bibliographystyle{apalike}

\end{document}
