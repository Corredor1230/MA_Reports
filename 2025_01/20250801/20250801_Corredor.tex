\documentclass[a4paper, 12pt]{article}
%\usepackage[utf8]{inputenc}  % Ensures proper encoding for non-Japanese text
\usepackage{xeCJK}           % Allows Japanese characters
\usepackage{geometry}
\usepackage{graphicx}
\usepackage{hyperref}
\usepackage{caption}
\usepackage{subcaption}

\geometry{margin=1in}

\setCJKmainfont{MS Mincho}  
\title{進捗報告}
\author{コレドール・パブロ}
\date{2025年8月01日}

\begin{document}

\maketitle

\section*{近況}
\begin{itemize}
    \item インターンシップに入れませんでした!(日本語能力のせいか、ただ本研究に適合していなかったか)
    \item 先週の木曜日から関西にいて万博にも行ってきました。
    \begin{itemize}
        \item トルクメニスタン、ウズベキスタン、UAEなどとにかくヨーロッパ・西洋じゃない国のパビリオンを訪ねてみました。
        \item やはり気になり、コロンビア、イタリアのパビリオンにも行きました。コロンビアの料理も久しぶりに食べられました!
    \end{itemize}
    \item レポート、スタジオ検定、発表などの準備も同時に何とかできました。
    \item とある東京の会社と最近よくコラボレーションしていて、日本のアーティストの歌詞を訳しています。
    \item コロンビアの会社からの連絡が来て、日本語でのオーディオブックを編集(バイト)できるエンジニアを探していたらしいので、とりあえず応募しました。
\end{itemize} 

\section*{やったこと}
\begin{itemize}
    \item 楽器音響学のプレゼンのために、ギターのデータベースを分析した。
\end{itemize}

\section*{考えていること}

このような一般的な分析は論文を書くときに役に立つので、マンドリンのデータの前処理を早速終わらせて、分析する。

% \begin{figure}
% \centering
% \begin{subfigure}{1.0\textwidth}
%   \centering
%   \includegraphics[width=0.9\linewidth]{../media/training_data/20250613/film_epoch320_pitch.png}
%   \label{fig:0606modelWave}
% \end{subfigure}
% \vspace{5mm}
% \begin{subfigure}{1.0\textwidth}
%   \centering
%   \includegraphics[width=0.9\linewidth]{../media/training_data/20250613/film_epoch320_spectro.png}
%   \label{fig:0606modelSpectro}
% \end{subfigure}
% \caption{ZCR}
% \label{fig:0606Model}
% \end{figure}

% \begin{figure}
% \centering
% \begin{subfigure}{1.0\textwidth}
%   \centering
%   \includegraphics[width=0.9\linewidth]{../media/training_data/20250613/film_epoch320_pitch.png}
%   \label{fig:0606modelWave}
% \end{subfigure}
% \vspace{5mm}
% \begin{subfigure}{1.0\textwidth}
%   \centering
%   \includegraphics[width=0.9\linewidth]{../media/training_data/20250613/film_epoch320_spectro.png}
%   \label{fig:0606modelSpectro}
% \end{subfigure}
% \caption{Rolloff}
% \label{fig:0606Model}
% \end{figure}

% \begin{figure}
% \centering
% \begin{subfigure}{1.0\textwidth}
%   \centering
%   \includegraphics[width=0.9\linewidth]{../media/training_data/20250613/film_epoch320_pitch.png}
%   \label{fig:0606modelWave}
% \end{subfigure}
% \vspace{5mm}
% \begin{subfigure}{1.0\textwidth}
%   \centering
%   \includegraphics[width=0.9\linewidth]{../media/training_data/20250613/film_epoch320_spectro.png}
%   \label{fig:0606modelSpectro}
% \end{subfigure}
% \caption{ZCR}
% \label{fig:0606Model}
% \end{figure}

% \begin{figure}
% \centering
% \begin{subfigure}{1.0\textwidth}
%   \centering
%   \includegraphics[width=0.9\linewidth]{../media/training_data/20250613/elec_pitch.png}
%   \caption{トレーニングデータとして使用しているギターサンプル}
%   \label{fig:eGuitWave}
% \end{subfigure}%
% \vspace{5mm}
% \begin{subfigure}{1.0\textwidth}
%   \centering
%   \includegraphics[width=0.9\linewidth]{../media/training_data/20250613/elec_spectro.png}
%   \caption{そのスペクトルグラム}
%   \label{fig:eGuitSpectro}
% \end{subfigure}
% \caption{エレキギター}
% \label{fig:elecGuit}
% \end{figure}


\section*{やること}
\begin{itemize}
    \item 来週からコーディングに集中するつもりです!
    \item 自分のMIDIコントローラーを作ります(パーツはもう買いました)
    \item 日本語での自己表現力を増やす方法を検討する。
\end{itemize}

%Bibliography (duh)
%\bibliography{../ref}
%\bibliographystyle{apalike}

\end{document}