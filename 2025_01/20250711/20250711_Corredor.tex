\documentclass[a4paper, 12pt]{article}
%\usepackage[utf8]{inputenc}  % Ensures proper encoding for non-Japanese text
\usepackage{xeCJK}           % Allows Japanese characters
\usepackage{geometry}
\usepackage{graphicx}
\usepackage{hyperref}
\usepackage{caption}
\usepackage{subcaption}

\geometry{margin=1in}

\setCJKmainfont{MS Mincho}  
\title{進捗報告}
\author{コレドール・パブロ}
\date{2025年7月11日}

\begin{document}

\maketitle

\section*{近況}
\begin{itemize}
    \item パソコンはまだ生きている!今はゾンビー形
\end{itemize} 

\section*{やったこと}
\begin{itemize}
    \item 周期スプリッターのプログラムに集中した。
    \item GUIやcsv製作の昨日も書いた。
    \item これで角倍音のエンベロープを分析でき、位相の変わりも含まれ、モデルのトレーニングができるのだろう。
\end{itemize}

\section*{考えていること}

\cite{sitranoPiano}によると、Sine, Noise, Transientに基づくモデルのトレーニングが楽器の生成のために使用できるため、この道で続く。

周期から周波数や位相の情報を得るために、各周期をHanning窓を使って、ループして、そこからFFTで取れるかと思う。

論文を読んでもまだ、音源の前処理は完全に理解できないため、それに集中しなければならない。

% \begin{figure}
% \centering
% \begin{subfigure}{1.0\textwidth}
%   \centering
%   \includegraphics[width=0.9\linewidth]{../media/training_data/20250613/film_epoch320_pitch.png}
%   \caption{現在使用しているモデルの出力}
%   \label{fig:0606modelWave}
% \end{subfigure}
% \vspace{5mm}
% \begin{subfigure}{1.0\textwidth}
%   \centering
%   \includegraphics[width=0.9\linewidth]{../media/training_data/20250613/film_epoch320_spectro.png}
%   \caption{そのスペクトルグラム}
%   \label{fig:0606modelSpectro}
% \end{subfigure}
% \caption{モデル}
% \label{fig:0606Model}
% \end{figure}

% \begin{figure}
% \centering
% \begin{subfigure}{1.0\textwidth}
%   \centering
%   \includegraphics[width=0.9\linewidth]{../media/training_data/20250613/elec_pitch.png}
%   \caption{トレーニングデータとして使用しているギターサンプル}
%   \label{fig:eGuitWave}
% \end{subfigure}%
% \vspace{5mm}
% \begin{subfigure}{1.0\textwidth}
%   \centering
%   \includegraphics[width=0.9\linewidth]{../media/training_data/20250613/elec_spectro.png}
%   \caption{そのスペクトルグラム}
%   \label{fig:eGuitSpectro}
% \end{subfigure}
% \caption{エレキギター}
% \label{fig:elecGuit}
% \end{figure}


\section*{やること}
\begin{itemize}
    \item 音源から必要な情報を取る方法を探します。
    \item 間に合えば、中間発表で見せられるモデルをトレーニングします。
\end{itemize}

%Bibliography (duh)
\bibliography{../ref}
\bibliographystyle{apalike}

\end{document}