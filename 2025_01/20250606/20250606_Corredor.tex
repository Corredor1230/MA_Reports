\documentclass[a4paper, 12pt]{article}
%\usepackage[utf8]{inputenc}  % Ensures proper encoding for non-Japanese text
\usepackage{xeCJK}           % Allows Japanese characters
\usepackage{geometry}
\usepackage{graphicx}
\usepackage{hyperref}
\usepackage{caption}
\usepackage{subcaption}

\geometry{margin=1in}

\setCJKmainfont{MS Mincho}  % You can change this to a different Japanese font if needed

\title{進捗報告}
\author{コレドール・パブロ}
\date{2025年6月6日}

\begin{document}

\maketitle

\section*{近況}
% Write your recent findings here.
\begin{itemize}
    \item 翻訳の依頼が増えてきて忙しい。
    \item 自分の音楽制作に集中し、動画制作・編集もしていた。
    \item 天気がいい間(まだ梅雨が始まっていない、気温もまだ30度以下)、よく散歩していたり、公園でギターを弾いたりしている。
    \item 前学期に作っていた周期スプリッターの開発に戻った。本研究で使用しなくても、高調波信号の分析に便利だと思う
\end{itemize} 

\section*{やったこと}
\begin{itemize}
    \item 線形モデルを畳み込みモデルにした。結果が若干改良されたが、まだ使えそうもない。
    \item 今まではスペクトルグラムの分析に頼り、モデルが周波数を理解するだろうと仮説したうえで、トレーニングしていたものの、周波数の認識が低いため、メタデータとして各サンプルの周波数を追加してみたら、変化無し。 
\end{itemize}

\section*{考えていること}

周波数を認識するモデルの開発は意外と難しくなっているものの、毎週進捗し、出力も遅くてもどんどん希望される結果に近づいているため、続くしかありません。ここからは開発計画(変化の記録など)を大事にして、進む予定。

前回の位相が含まれているモデルに戻って、現在の畳み込みモデルの出力はもしかして精度が高いかもしれない。

\begin{figure}
\centering
\begin{subfigure}{1.0\textwidth}
  \centering
  \includegraphics[width=0.9\linewidth]{../media/training_data/20250606/waveform_epoch30.png}
  \caption{現在使用しているモデルの出力}
  \label{fig:0606modelWave}
\end{subfigure}
\vspace{5mm}
\begin{subfigure}{1.0\textwidth}
  \centering
  \includegraphics[width=0.9\linewidth]{../media/training_data/20250606/spectrogram_epoch30.png}
  \caption{そのスペクトルグラム}
  \label{fig:0606modelSpectro}
\end{subfigure}
\caption{モデル}
\label{fig:0606Model}
\end{figure}

\begin{figure}
\centering
\begin{subfigure}{1.0\textwidth}
  \centering
  \includegraphics[width=0.9\linewidth]{../media/training_data/20250606/waveform_elec329.png}
  \caption{トレーニングデータとして使用しているギターサンプル}
  \label{fig:eGuitWave}
\end{subfigure}%
\vspace{5mm}
\begin{subfigure}{1.0\textwidth}
  \centering
  \includegraphics[width=0.9\linewidth]{../media/training_data/20250606/spectrogram_elec329.png}
  \caption{そのスペクトルグラム}
  \label{fig:eGuitSpectro}
\end{subfigure}
\caption{エレキギター}
\label{fig:elecGuit}
\end{figure}


\section*{やること}
% Outline future steps here.
\begin{itemize}
    \item \\(周波数の問題をデバッグしてみる。)^2
\end{itemize}

% Add your bibliography here.
%\bibliography{../ref}
%\bibliographystyle{apalike}

\end{document}