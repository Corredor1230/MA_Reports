\documentclass[a4paper, 12pt]{article}
%\usepackage[utf8]{inputenc}  % Ensures proper encoding for non-Japanese text
\usepackage{xeCJK}           % Allows Japanese characters
\usepackage{geometry}
\usepackage{graphicx}
\usepackage{hyperref}
\usepackage{caption}
\usepackage{subcaption}

\geometry{margin=1in}

\setCJKmainfont{MS Mincho}  
\title{進捗報告}
\author{コレドール・パブロ}
\date{2025年6月27日}

\begin{document}

\maketitle

\section*{近況}
\begin{itemize}
    \item 先週、読み始めた「イスラームの歴史」についての本は、最初の80ページ研究方法について述べて、まだ歴史そのもあまり触れていない。
    \item 東京国際交流館経由、墨田川高等学校の国際交流祭りみたいなイベントに誘われて、自国についての発表(?)決定。
\end{itemize} 

\section*{やったこと}
\begin{itemize}
    \item ピッチのある出力ができました!
    \item またDDSP(Differentiable Digital Signal Processing)\cite{DDSP_Traditional}に向けて、入力データの倍音を使ってモデルをトレーニングしてみたら、結果がかなり改善できた。
    \item マンドリンの収録されているサンプルの前処理を実行し始め、新しいサンプルも収録中です。
\end{itemize}

\section*{考えていること}

今週生成できた出力のピッチが明瞭に聞こえるが、一方アタックの方が悪化した。その結果は予想外ではないので、次はアタックとサステインを異なるモデルで生成してみる。現在用いているモデルでは、アタックがスペクトルの平均値、または最も音量が大きい周波数として生成されているため、撥弦楽器のアイデンティティが弱い。

% \begin{figure}
% \centering
% \begin{subfigure}{1.0\textwidth}
%   \centering
%   \includegraphics[width=0.9\linewidth]{../media/training_data/20250613/film_epoch320_pitch.png}
%   \caption{現在使用しているモデルの出力}
%   \label{fig:0606modelWave}
% \end{subfigure}
% \vspace{5mm}
% \begin{subfigure}{1.0\textwidth}
%   \centering
%   \includegraphics[width=0.9\linewidth]{../media/training_data/20250613/film_epoch320_spectro.png}
%   \caption{そのスペクトルグラム}
%   \label{fig:0606modelSpectro}
% \end{subfigure}
% \caption{モデル}
% \label{fig:0606Model}
% \end{figure}

% \begin{figure}
% \centering
% \begin{subfigure}{1.0\textwidth}
%   \centering
%   \includegraphics[width=0.9\linewidth]{../media/training_data/20250613/elec_pitch.png}
%   \caption{トレーニングデータとして使用しているギターサンプル}
%   \label{fig:eGuitWave}
% \end{subfigure}%
% \vspace{5mm}
% \begin{subfigure}{1.0\textwidth}
%   \centering
%   \includegraphics[width=0.9\linewidth]{../media/training_data/20250613/elec_spectro.png}
%   \caption{そのスペクトルグラム}
%   \label{fig:eGuitSpectro}
% \end{subfigure}
% \caption{エレキギター}
% \label{fig:elecGuit}
% \end{figure}


\section*{やること}
\begin{itemize}
    \item トレーニング続きます。特にアタックとサステインのモデル。
    \item 新しいモデルにGANなどの使用を検討する。
\end{itemize}

%Bibliography (duh)
\bibliography{../ref}
\bibliographystyle{apalike}

\end{document}