\documentclass[a4paper, 12pt]{article}
%\usepackage[utf8]{inputenc}  % Ensures proper encoding for non-Japanese text
\usepackage{xeCJK}           % Allows Japanese characters
\usepackage{geometry}
\usepackage{graphicx}
\usepackage{hyperref}

\geometry{margin=1in}

\setCJKmainfont{MS Mincho}  % You can change this to a different Japanese font if needed

\title{進捗報告}
\author{コレドール・パブロ}
\date{2024年}

\begin{document}

\maketitle

\section*{近況}
% Write your recent findings here.
\begin{itemize}
    \item 
\end{itemize} 

\section*{やったこと}
\begin{itemize}
    \item \cite{latentSpaceInterpolation}を拝読した。
\end{itemize}

\section*{考えていること}

モデルのトレーニングの計画を建て、二つの楽器の分析をするモデルをテストしてみる。
そのためにまずはVACを使用し、楽器のデータベースをLatent Space(特性要素のベクトル)にする。\cite{latent_space}

二つの楽器のベクトルを補間し、DDSPでデコードし、出力を生成できるモデルが完成する。

トレーニングされたモデルをC++で使用できるライブラリとしてエキスポートし、JUCEのプラグインに含む。
モデルのエキスポートについて調べてみたら、ONNXというフォーマットが出てきて、TensorFlowやPyTorchのVAC
が対応されているため、使用しようと思っている。

今週、始めてのトレーニングが完成したかったのに、前処理が長くなって結局間に合わなかったが、
Pythonのコードはほぼ出来ているため、もうそろそろモデルのテストを始まれそうです。

\section*{やること}
% Outline future steps here.
\begin{itemize}
    \item モデルのテストとONNXのエキスポートをしてみます。
    \item 国では楽器がまだ始まっていないので、とりあえずそれを待ち、一か月後ぐらい
    ハープの演奏者にまた連絡します。
\end{itemize}

% Add your bibliography here.
\bibliography{../ref}
\bibliographystyle{apalike}

\end{document}