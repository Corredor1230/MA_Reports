\documentclass[a4paper, 12pt]{article}
%\usepackage[utf8]{inputenc}  % Ensures proper encoding for non-Japanese text
\usepackage{xeCJK}           % Allows Japanese characters
\usepackage{geometry}
\usepackage{graphicx}
\usepackage{hyperref}

\geometry{margin=1in}

\setCJKmainfont{MS Mincho}  % You can change this to a different Japanese font if needed

\title{進捗報告}
\author{コレドール・パブロ}
\date{2025年5月16日}

\begin{document}

\maketitle

\section*{近況}
% Write your recent findings here.
\begin{itemize}
    \item 夏が来る。あつくて、恐い。11月まで人間になれない。
\end{itemize} 

\section*{やったこと}
\begin{itemize}
    \item CMakeでやっとLibtorchをプログインに組み込むことができた(動的ライブラリとしてだけど)!
    \item モデルからの出力を音にする方法を検討している。
    \item \cite{FIRNet}という文献を見つけて、もしかしてちょうど探していることかと思っていた。
\end{itemize}

\section*{考えていること}

撥弦楽器のサンプルの短時間フーリエ変換またはメルスペクトルグラム(等級)と位相情報、そして楽器の指数、弦の指数、ダイナミクスの指数から生成されるLatent Spaceをエンコードし、トレーニングデータとして使用する。モデルの基本的な動作はLatent Spaceから等級と位相情報を再現する。その出力をまた音にする段階は特に複雑なので、現在は方法検討中である。恐らくFFTW(Fastest Fourier Transform in the West)を動的ライブラリとして組み込み、モデルの出力から逆短時間フーリエ変換を行って、オーディオを生成できるかと思う。FFTWは軽く、速く、使用した経験もあるため、最も相応しい方法だと思う。

しかし、FIRNet\cite{FIRNet}の方は精度が高く、使い方もよく記録されているため、ボコーダーに基づいている方法も検討している。

\section*{やること}
% Outline future steps here.
\begin{itemize}
    \item iSTFTのコードを書く。
    \item FIRNetの文献を読み終わる。
\end{itemize}

\section{}

% Add your bibliography here.
\bibliography{../ref}
\bibliographystyle{apalike}

\end{document}