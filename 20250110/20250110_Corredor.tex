\documentclass[a4paper, 12pt]{article}
%\usepackage[utf8]{inputenc}  % Ensures proper encoding for non-Japanese text
\usepackage{xeCJK}           % Allows Japanese characters
\usepackage{geometry}
\usepackage{graphicx}
\usepackage{hyperref}

\geometry{margin=1in}

\setCJKmainfont{MS Mincho}  % You can change this to a different Japanese font if needed

\title{進捗報告}
\author{コレドール・パブロ}
\date{2025年01月10日}

\begin{document}

\maketitle

\section*{近況}
% Write your recent findings here.
\begin{itemize}
    \item 年末はちゃんと休んで、自分の趣味に集中したり、
    友達に会ったりしていました。
    \item やはりクリスマス・年末時期に帰国しなくて
    少し寂しかったですが、天気がずっと良かったですから
    色々な場所に行ったり、散歩したりして楽しかったです。
    \item 2025年度修士を始まり、頑張ります。
\end{itemize} 

\section*{やったこと}
\begin{itemize}
    \item ちゃんと休み、あまり研究に集中しないつもりでしたが、
    結局信号の周期の自動編集プログラムが楽しくて、大体動く
    プログラムが開発できました。
    \item マンドリンの収録を始めましたが、まだまだかかりそうです。
    \item 弦楽器の振動と物理的な特性の文献を読み始めました。\cite{String_vibration}
\end{itemize}

\section*{考えていること}

1コースに2弦があるマンドリンの場合には弦1本1本を収録せず、
コースを収録していますが、そうすると、コース1弦の楽器と
全然違う波形になるので、分析の時は問題になるかもしれません。

周期の認識は現在のプログラムでかなり良くなりましたが、
ずれていく問題もあるため、まだ完璧ではない。しかし、
このデータではADSRの情報やスペクトル成分の変わりも
確かめるかと思うので、このプログラムは使えそうです。

\section*{やること}
% Outline future steps here.
\begin{itemize}
    \item 収録、研究、開発を続く
\end{itemize}

% Add your bibliography here.
\bibliography{../ref}
\bibliographystyle{apalike}

\end{document}