\documentclass[a4paper, 12pt]{article}
%\usepackage[utf8]{inputenc}  % Ensures proper encoding for non-Japanese text
\usepackage{xeCJK}           % Allows Japanese characters
\usepackage{geometry}
\usepackage{graphicx}
\graphicspath{{../media/}}
\usepackage{hyperref}

\geometry{margin=1in}

\setCJKmainfont{MS Mincho}  % You can change this to a different Japanese font if needed

\title{進捗報告}
\author{コレドール・パブロ}
\date{2024年11月8日}

\begin{document}

\maketitle

\section*{近況}
% Write your recent findings here.
\begin{itemize}
    \item 先週の週末は風邪をひいた
    \item 演習の発表を完了(?)した
    \item オープンソースプロジェクトに関わったり、Discordのサーバーに入ったりして、
    CMakeはやっぱり難しい
    \item PlugDataが認識できるテストオブジェクトを開発してみた
\end{itemize} 

\section*{やったこと}
\begin{itemize}
    \item DDSPの勉強は問題なく進んでいる
    \item 別の参考を探してみたら、\cite{ddspTutorial}
    で進捗している。
    \item DDSPのGithubにもTrainingのチュートリアルなどもあるので、
    ここからは参考として使用して、自分のモデルのトレーニングができるだろうと
    思っている。
\end{itemize}

\section*{考えていること}

現在読んでいる文献のおかげで、次々のステップは大体理解できたと思うので、
ここからはいくつかの開発実験を行って、トレーニングの結果によって、次の計画を
建てると考えている。

\includegraphics[scale=0.2]{ddspoverview}

\section*{やること}
% Outline future steps here.
\begin{itemize}
    \item 容易なモデルをトレーニングしてみる
    \item 来週の草加ハープフェスティバルに行く
\end{itemize}

% Add your bibliography here.
\bibliography{../ref}
\bibliographystyle{apalike}

\end{document}