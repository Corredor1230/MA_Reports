\documentclass[a4paper, 12pt]{article}
%\usepackage[utf8]{inputenc}  % Ensures proper encoding for non-Japanese text
\usepackage{xeCJK}           % Allows Japanese characters
\usepackage{geometry}
\usepackage{graphicx}
\usepackage{hyperref}

\geometry{margin=1in}

\setCJKmainfont{MS Mincho}  % You can change this to a different Japanese font if needed

\title{進捗報告}
\author{コレドール・パブロ}
\date{2024年11月29日}

\begin{document}

\maketitle

\section*{近況}
% Write your recent findings here.
\begin{itemize}
    \item インフルエンザ予防接種を受けた(嬉)
    \item 新しい食器棚を買ったので、部屋の整理を徹底的にやり直した。
    \item 年金免除の手続きを完了した
\end{itemize} 

\section*{やったこと}
\begin{itemize}
    \item Variational AutoencoderとDDSPのチュートリアルを見つけて\cite{autoencoder_tutorial}、やっと
    DDSPでの開発を開始した。今までは音量の変更を認識出来るプログラムや
    二つの信号を比較できるプログラムを開発して、Timbre Interpolationのも
    開発し始めた。
    \item 知っている元・同級生、知り合い、友達などと話し合ったり、収録できる方を探し始めた。
\end{itemize}

\section*{考えていること}

カープラスストロングや物理モデルのシンセサイザーも訓練のデータベースに入れるかどうか
少し悩んでいる。そのデジタル楽器は撥弦楽器と似た特性もあるものの、本物の弦ではないため、
研究にどのようが影響になるか気になる。\\~\\
現在収録できそうな楽器:
\begin{itemize}
    \item 三味線
    \item マンドリン
    \item アコギター(収録中)
    \item エレキギター(収録完了)
    \item エレキバス
    \item チャランゴ
    \item ウクレレ
    \item バンドゥリア
    \item クアト\\
\end{itemize}

現在収録したいけど演奏者を知らない楽器:
\begin{itemize}
    \item リラ
    \item バラライカ
    \item ハープ
    \item 琴
    \item リュート
    \item テオルボ
    \item バンジョー\\
\end{itemize}

収録するべきかどうか不明:
\begin{itemize}
    \item カープラスストロング・物理モデルシンセサイザー
    \item サントゥール
    \item チェンバロ
    \item コントラバス(ジャズなど)
    \item ピッツィカートのバイオリン等\\
\end{itemize}
もちろん、サンプルのライブラリーも使えるけれど、サンプルの数、またダイナミックの
ステージ等はかなり自分のシステムと異なるため、出力のバランスが崩れるかもしれない。

\section*{やること}
\begin{itemize}
    \item Timbre Interpolationができるプログラムを完成する。
    \item 収録続ける
\end{itemize}

% Add your bibliography here.
\bibliography{../ref}
\bibliographystyle{apalike}

\end{document}