\documentclass[a4paper, 12pt]{article}
%\usepackage[utf8]{inputenc}  % Ensures proper encoding for non-Japanese text
\usepackage{xeCJK}           % Allows Japanese characters
\usepackage{geometry}
\usepackage{graphicx}
\usepackage{hyperref}

\geometry{margin=1in}

\setCJKmainfont{MS Mincho}  % You can change this to a different Japanese font if needed

\title{進捗報告}
\author{コレドール・パブロ}
\date{2024年10月25日}

\begin{document}

\maketitle

\section*{近況}
% Write your recent findings here.
\begin{itemize}
    \item テオドール・アドルノの「啓蒙の弁証法」を読み始め、色々考えている。
    \item 容赦ないの暑い天気がようやく許しを与え、また長い散歩に出かけている。
\end{itemize} 

\section*{やったこと}
\begin{itemize}
    \item 前回の2017年の筆者のJesse EngelはDDSPというプロジェクトの担当の方だと
    調べて、二つの2020年\cite{DDSP} と2022年\cite{DDSP2022} の文献を見つけて読みました。
    \item \cite{DDSP2022}を読んだが、やはり不明な概念が多すぎるため、
    それらをリストに書き、一つずつの定義を確認している。
    \begin{itemize}
        \item Strided Convolution (ストライド畳み込み)
        \item Teacher forcing (教師強制)
        \item Automatic differentiation (自動微分)
        \item Deterministic autoencoder (決定論的オートエンコーダー)
        \item Adversarial training (敵対的訓練)
        \item Jacobian design (ヤコビアン設計)
        \item Stochastic latents (確率的潜在変数)
        \item CREPE model
    \end{itemize}
    
\end{itemize}

\section*{考えていること}
DDSPの文献を読んだら、二つの大事なことに気づいた。

一つは自分の機械学習についての知識の足りなさである。理解できない専用語が山ほどありながら、
研究を行うのも難しくなるに違いないので、その分野の基礎的な概念を勉強する必要があると
思う。

もう一つはこの研究を行うためにはゼロから開発する必要があるコードの量は意外と少ないということである。
やはりDDSPの文献を発表した研究者はもうすでに役に立つライブラリーをGithubにアップロードして、
OpenSourceであるため、すぐに利用できると思う。

モデルのトレーニングコードやチュートリアルも全てレポシトリーに載せているので、
楽器のサンプルさえあれば、すぐトレーニング始められる状態である。

\section*{やること}
% Outline future steps here.
\begin{itemize}
    \item 
\end{itemize}

% Add your bibliography here.
\bibliography{../ref}
\bibliographystyle{apalike}

\end{document}