\documentclass[a4paper, 12pt]{article}
%\usepackage[utf8]{inputenc}  % Ensures proper encoding for non-Japanese text
\usepackage{xeCJK}           % Allows Japanese characters
\usepackage{geometry}
\usepackage{graphicx}
\usepackage{hyperref}

\geometry{margin=1in}

\setCJKmainfont{MS Mincho}  % You can change this to a different Japanese font if needed

\title{進捗報告}
\author{コレドール・パブロ}
\date{2025年}

\begin{document}

\maketitle

\section*{近況}
% Write your recent findings here.
\begin{itemize}
    \item 引越が決定されたので、少しストレス中。
    \item 母親は3月訪ねて来たので、忙しかった。
    \item TeensyとRaspberry Piを買って、色々勉強している。
\end{itemize} 

\section*{やったこと}
\begin{itemize}
    \item 学部の大学のAESの発表をして、そのためにいくつかの面白い実験をした。
    \item ハープの収録の計画をやっと建てている。
\end{itemize}

\section*{考えていること}

Teensyを使って何か面白いことを作りたいと思う。開発しているシンセに関するかどうか
まだ不明だが、役に立つかと思うので、ここから少し考えていく。

トレーニングのためにColabを使おうと思っていて、その使い方を調べ始めた。

ChatGPT、AI、機械学習に対しての意見が最近どんどん強くなってきて、この研究にも影響するかもしれない。

最近はデジタル楽器のアイデンティティ、「シンセサイザーの音」という概念等について考えている。

\section*{やること}
% Outline future steps here.
\begin{itemize}
    \item いつも通り続きます。
    \item 機械学習・AIの倫理についてもっと知りたいと思います。
\end{itemize}

% Add your bibliography here.

\end{document}