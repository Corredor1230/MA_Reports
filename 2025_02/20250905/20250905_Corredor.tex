\documentclass[a4paper, 12pt]{article}
%\usepackage[utf8]{inputenc}  % Ensures proper encoding for non-Japanese text
\usepackage{xeCJK}           % Allows Japanese characters
\usepackage{geometry}
\usepackage{graphicx}
\usepackage{hyperref}
\usepackage{caption}
\usepackage{subcaption}

\geometry{margin=1in}

\setCJKmainfont{MS Mincho}  
\title{進捗報告}
\author{コレドール・パブロ}
\date{2025年9月05日}

\begin{document}

\maketitle

\section*{近況}
\begin{itemize}
    \item 弟が無事に帰りました。
    \item 8月は基本的に夏休みでした!これから9月は開発へ!
    \item 次回のインターンシップの申し込みの準備しています。
\end{itemize} 

\section*{やったこと}
\begin{itemize}
    \item 加算合成シンセサイザーのコントロールを検討している。
\end{itemize}

\section*{考えていること}

生信号を生成するモデルのトレーニングは複雑で、結果も期待通りに出力できていなかったため、今は音源の次元削減を検討し、加算合成シンセサイザーをコントロールする信号のデータでモデルをトレーニングすれば・・・?と検討している。

\cite{latentSpaceInterpolation}においてはYAMAHA DX7のFMパラメーターに基づき、モデルをトレーニングできた研究もあるため、今度は音源からのパラメーターでモデルをトレーニングし、加算合成シンセサイザーで生成してみます。その加算合成シンセサイザーに\cite{sitranoPiano}に出てくる分析方法を加えたらもしかしてアタックもノイズも生成できるかと考えています。

% \begin{figure}
% \centering
% \begin{subfigure}{1.0\textwidth}
%   \centering
%   \includegraphics[width=0.9\linewidth]{../media/training_data/20250613/film_epoch320_pitch.png}
%   \label{fig:0606modelWave}
% \end{subfigure}
% \vspace{5mm}
% \begin{subfigure}{1.0\textwidth}
%   \centering
%   \includegraphics[width=0.9\linewidth]{../media/training_data/20250613/film_epoch320_spectro.png}
%   \label{fig:0606modelSpectro}
% \end{subfigure}
% \caption{ZCR}
% \label{fig:0606Model}
% \end{figure}

% \begin{figure}
% \centering
% \begin{subfigure}{1.0\textwidth}
%   \centering
%   \includegraphics[width=0.9\linewidth]{../media/training_data/20250613/film_epoch320_pitch.png}
%   \label{fig:0606modelWave}
% \end{subfigure}
% \vspace{5mm}
% \begin{subfigure}{1.0\textwidth}
%   \centering
%   \includegraphics[width=0.9\linewidth]{../media/training_data/20250613/film_epoch320_spectro.png}
%   \label{fig:0606modelSpectro}
% \end{subfigure}
% \caption{Rolloff}
% \label{fig:0606Model}
% \end{figure}

% \begin{figure}
% \centering
% \begin{subfigure}{1.0\textwidth}
%   \centering
%   \includegraphics[width=0.9\linewidth]{../media/training_data/20250613/film_epoch320_pitch.png}
%   \label{fig:0606modelWave}
% \end{subfigure}
% \vspace{5mm}
% \begin{subfigure}{1.0\textwidth}
%   \centering
%   \includegraphics[width=0.9\linewidth]{../media/training_data/20250613/film_epoch320_spectro.png}
%   \label{fig:0606modelSpectro}
% \end{subfigure}
% \caption{ZCR}
% \label{fig:0606Model}
% \end{figure}

% \begin{figure}
% \centering
% \begin{subfigure}{1.0\textwidth}
%   \centering
%   \includegraphics[width=0.9\linewidth]{../media/training_data/20250613/elec_pitch.png}
%   \caption{トレーニングデータとして使用しているギターサンプル}
%   \label{fig:eGuitWave}
% \end{subfigure}%
% \vspace{5mm}
% \begin{subfigure}{1.0\textwidth}
%   \centering
%   \includegraphics[width=0.9\linewidth]{../media/training_data/20250613/elec_spectro.png}
%   \caption{そのスペクトルグラム}
%   \label{fig:eGuitSpectro}
% \end{subfigure}
% \caption{エレキギター}
% \label{fig:elecGuit}
% \end{figure}


\section*{やること}
\begin{itemize}
    \item 加算合成シンセサイザーのパラメーターとそのほかのノイズ、アタックを検討します。
    \item 開発に一生懸命頑張ります。
    \item インターンシップの応募を完了します。
\end{itemize}

%Bibliography (duh)
\bibliography{../ref}
\bibliographystyle{apalike}

\end{document}