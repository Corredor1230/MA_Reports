\documentclass[a4paper, 12pt]{article}
%\usepackage[utf8]{inputenc}  % Ensures proper encoding for non-Japanese text
\usepackage{xeCJK}           % Allows Japanese characters
\usepackage{geometry}
\usepackage{graphicx}
\usepackage{hyperref}

\geometry{margin=1in}

\setCJKmainfont{MS Mincho}  % You can change this to a different Japanese font if needed

\title{進捗報告}
\author{コレドール・パブロ}
\date{2025年10月3日}

\begin{document}

\maketitle

\section*{近況}
% Write your recent findings here.
\begin{itemize}
    \item Hollow Knight: Silksongが出てきて、圧倒的に楽しい。
    \item イタリア語の集中講義を受けました。面白かったです。
    \item 今週、都立小川高等学校の国際交流イベントでプレゼンしました。
\end{itemize} 

\section*{やったこと}
\begin{itemize}
    \item 楽音信号の倍音の分析から出力ができ、ピッチも問題なくコントロールできる!音色も大体ギターに近くなった!\cite{sitrano}
    \item コードをきれいに書き直し、用途が広い分析プログラムにした。
\end{itemize}

\section*{考えていること}

トランジェントの分析・生成方法を検討している。ノイズの分析にはいくつかの方法を見つけたものの、トランジェントは非常に重要なので、十分な時間を使用し、相応しい方法を見つける。

現在使っている倍音エンベロープの分析方法:

\begin{itemize}
    \item 信号のトランジェントの初サンプルを探し、1つのFFT窓後の振幅のピークを見つける。
    \item ピークのすぐ前のゼロクロスを見つける。
    \item 相互相関分析・ピッチにより、各周期の初サンプルを見つけ、ベクトルに登録する。
    \item 信号のエネルギーが安定になったサステインからのやや長いFFTを行って、最も重要な32個の周波数を見つけ、登録する。
    \item FFTウィンドウを埋めるために2周期のフレームを繰り返す。
    \item FFTウィンドウにHanning窓の処理を実施する。
    \item 重要な32個の周波数の大きさを登録する。1e-6の以下のビンは無視。
    \item 各ビンのエンベロープをフィルターし、csvファイルとしてエキスポートする。
\end{itemize}

倍音のエンベロープの分析方法についてのテストは今まで行っていないけれど、Hanning無し、周期のゼロクロスを使わず(普通のFFT)の分析などのテストが必要。

\begin{figure}
\centering
\includegraphics[width = 0.9\linewidth]{../../media/2025_02/EGuit_Harmonics_82Hz.png}
\caption{82Hzのエレキギターの倍音エンベロープ}
\label{fig:eGuitHarmonics}
\end{figure}

\begin{figure}
\centering
\includegraphics[width = 0.9\linewidth]{../../media/2025_02/AGuit_Harmonics_146.png}
\caption{146Hzのアコギターの倍音エンベロープ}
\label{fig:aGuitHarmonics}
\end{figure}

% \begin{figure}
% \centering
% \begin{subfigure}{1.0\textwidth}
%   \centering
%   \includegraphics[width=0.9\linewidth]{../media/training_data/20250613/elec_pitch.png}
%   \caption{トレーニングデータとして使用しているギターサンプル}
%   \label{fig:eGuitWave}
% \end{subfigure}%
% \vspace{5mm}
% \begin{subfigure}{1.0\textwidth}
%   \centering
%   \includegraphics[width=0.9\linewidth]{../media/training_data/20250613/elec_spectro.png}
%   \caption{そのスペクトルグラム}
%   \label{fig:eGuitSpectro}
% \end{subfigure}
% \caption{エレキギター}
% \label{fig:elecGuit}
% \end{figure}

\section*{やること}
% Outline future steps here.
\begin{itemize}
    \item ノイズ(単音・倍音でない成分)の分析方法を検討し、出力を生成してみる。
    \item 倍音のエンベロープを従い、色々な音を生成してみる。
\end{itemize}

% Add your bibliography here.
\bibliography{../../ref}
\bibliographystyle{apalike}

\end{document}