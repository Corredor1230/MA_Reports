\documentclass[a4paper, 12pt]{article}
%\usepackage[utf8]{inputenc}  % Ensures proper encoding for non-Japanese text
\usepackage{xeCJK}           % Allows Japanese characters
\usepackage{geometry}
\usepackage{graphicx}
\usepackage{hyperref}
\usepackage{caption}
\usepackage{subcaption}

\geometry{margin=1in}

\setCJKmainfont{MS Mincho}  % You can change this to a different Japanese font if needed

\title{進捗報告}
\author{コレドール・パブロ}
\date{2025年5月30日}

\begin{document}

\maketitle

\section*{近況}
% Write your recent findings here.
\begin{itemize}
    \item 将来の計画、自分の道などについて色々な事を考えている。
    \item 8月頃に万博へ行く予定です。まだ予約していないけど・・・
\end{itemize} 

\section*{やったこと}
\begin{itemize}
    \item 出力ができた!!!(まだ精度が低いし、上手く動いていないかもしれないけど)
    \item Nicolasと一緒にコードを読んでみたら、改善できる所を見つけた。
    \item VAEについて様々なことを学んでいる。\cite{SPINVAE1} 
    
    \cite{latentSpaceInterpolation}
\end{itemize}

\section*{考えていること}

iSTFT(位相が含まれている)方法の出力を詳しく分析すると、音が撥弦楽器に近くても、波形は全くトレーニングデータと違って、恐らく生成の段階にエラーが発生している。それに従って、STFTのみでトレーニングし、生成の段階でGriffin Limを使用してみたら、出力がノイズに近いものの、波形がちゃんと撥弦楽器に似ているため、周波数のデータが失われているが、Griffin Limでもっと精度が高い出力が生成できるかと思う。

\begin{figure}
\centering
\begin{subfigure}{.5\textwidth}
  \centering
  \includegraphics[width=0.9\linewidth]{../media/wavTraining1.png}
  \caption{iSTFT(位相が含まれている)}
  \label{fig:gen1}
\end{subfigure}%
\begin{subfigure}{.5\textwidth}
  \centering
  \includegraphics[width=0.9\linewidth]{../media/wavTraining2.png}
  \caption{Griffin Lim}
  \label{fig:gen2}
\end{subfigure}
\caption{波形全体のイメージ}
\label{fig:generalComp}
\end{figure}

\begin{figure}
\centering
\begin{subfigure}{.5\textwidth}
  \centering
  \includegraphics[width=0.9\linewidth]{../media/train1Close.png}
  \caption{iSTFT}
  \label{fig:zoom1}
\end{subfigure}%
\begin{subfigure}{.5\textwidth}
  \centering
  \includegraphics[width=0.9\linewidth]{../media/train2Close.png}
  \caption{Griffin Lim}
  \label{fig:zoom2}
\end{subfigure}
\caption{波形(20msのみ)}
\label{fig:zoomComp}
\end{figure}


\section*{やること}
% Outline future steps here.
\begin{itemize}
    \item 周波数の問題をデバッグしてみる。
\end{itemize}

% Add your bibliography here.
\bibliography{../ref}
\bibliographystyle{apalike}

% Increase parameters, increase number of convolutional layers
% Penalize the model for high loss value
% Non-negative matrix factorization
% Maybe separate spectrogram into two different arrays one with frequencies and one with 
% Linear Predictive Coding to split sound sources

\end{document}