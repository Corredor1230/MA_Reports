\documentclass[a4paper, 12pt]{article}
%\usepackage[utf8]{inputenc}  % Ensures proper encoding for non-Japanese text
\usepackage{xeCJK}           % Allows Japanese characters
\usepackage{geometry}
\usepackage{graphicx}
\usepackage{hyperref}

\geometry{margin=1in}

\setCJKmainfont{MS Mincho}  % You can change this to a different Japanese font if needed

\title{進捗報告}
\author{コレドール・パブロ}
\date{2025年4月25日}

\begin{document}

\maketitle

\section*{近況}
% Write your recent findings here.
\begin{itemize}
    \item 履修登録完了!
    \item 音響心理研究法や高臨場感は学部の時全くしなかったので、登録して嬉しい。
\end{itemize} 

\section*{やったこと}
\begin{itemize}
    \item 楽器の「アイデンティティ」という言葉についての文献を検索してみたら、\cite{IdentityAugmenting}を見つけた。
    \item 始めてモデルのトレーニングが完了できた(エラーなし済んだ)
    \begin{itemize}
        \item どれぐらい使用できるのかまだ不明
    \end{itemize}
    \item CMakeを避けることができいので、まだ練習が必要。
    \begin{itemize}
        \item ちょうど今年、Rustを使ってJUCEのプロジェクトのCMakeファイルを管理できるパッケージがリリースされて、かなり易しくなって来た。
    \end{itemize}
    \item 先週のデータ拡張の文献は拝読中。
    \item 文献の日誌を書き始めた。
    \begin{itemize}
        \item 各論文の概略や新しい専用語等を記録して、論文を書くときには役に立つかと思う。
        \item 自分のコメントや自分の研究との関係も記録している。
    \end{itemize}
\end{itemize}

\section*{考えていること}

トレーニングされたモデルは上手く音が出せるなら、ここから色々な面白い実験をしようと思う。
例えば、楽器だけではなく、弾いた板や弾いた縄等の音を録音して、データセットに入れたらどうなるかやはり確かめたい。

今はメルスペクトルグラムとPytorchのNNを使っているものの、まだDDSPやオーディオに向いているライブラリーを使わないと判断していないため、今度の結果によって、別ので再度トレーニングする必要があるかもしれない。

\section*{やること}
% Outline future steps here.
\begin{itemize}
    \item トレーニングされたモデル(.pt)とJUCEの総合を解決してみる。
    \item 何か音が発生できれば、その特徴を確認したり、自分の客観的な印象を記録したりする。
    \item Epochの数を増やしたり、サンプルの情報(長さ、サンプリング周波数など)を変更したりしてみたい。
    \item マンドリンの前処理をし、データセットに入れる。
\end{itemize}

\section{}

% Add your bibliography here.
\bibliography{../ref}
\bibliographystyle{apalike}

\end{document}