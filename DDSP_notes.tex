\section{DDSP}

\cite{DDSP} used as main reference for the research.

\subsection{Important bullet points}

\begin{itemize}
    \item Interpretable and modular approach to generative modeling
    \item Classic signal processing elements with deep learning methods
    \item Models that rely on strided convolution or windowing (STFT)
    need to align waveshapes or suffer from spectral leakage.
    \item DDSP takes the approach of vocoders in using oscillators to
    synthesize signals.
    \item According to the DDSP paper \cite{DDSP}, the DDSP library is capable
    of extrapolating timbres not seen during training, and independent control
    over pitch and loudness during synthesis.
    \item References to Neural Source Filter \cite{neuralsourcefilter} imply
    this might be a good additional reference for research.

\end{itemize}

\subsection{Newly learned concepts from this document}

\begin{itemize}
    \item Strided Convolution:
    
    Convolution that has a hop length, meaning, it skips some information
    to avoid analyzing redundancies.

    \item Teacher forcing:
    
    Feeding back the correct answers into training algorithms 
    to reduce training times and lead the model in the right direction during training.

    \item Automatic differentiation:
    
    Also: 'algorithmic differentiation' or 'computational differentiation'
\end{itemize}

\subsection{How it relates to my research}