\documentclass[a4paper, 12pt]{article}
%\usepackage[utf8]{inputenc}  % Ensures proper encoding for non-Japanese text
\usepackage{xeCJK}           % Allows Japanese characters
\usepackage{geometry}
\usepackage{graphicx}
\usepackage{hyperref}

\geometry{margin=1in}

\setCJKmainfont{MS Mincho}  % You can change this to a different Japanese font if needed

\title{進捗報告}
\author{コレドール・パブロ}
\date{2024年11月1日}

\begin{document}

\maketitle

\section*{近況}
% Write your recent findings here.
\begin{itemize}
    \item 最近、Audacity4.0の開発についての記事を読んで、GitHubをCloneしてみたら、
    参加したいなと思ってきたので、暇な時間には少しいじって始めました。
\end{itemize} 

\section*{やったこと}
\begin{itemize}
    \item アコースティックギターの録音を始めました。
    \item DDSPのGithubからライブラリーをダウンロードし、テストし始めました。 \cite{DDSP}
    \item 撥弦楽器に集中する自分のプロジェクトと似ているシンセサイザーまたは研究文献を
    探し続けているものの、今までは見つけられないため、このまま進んでも大丈夫かと思ってきました。
    \item \cite{DDSP_Piano}においての研究は特にピアノのポリフォニーを生成できるシンセサイザーに
    焦点を当てている。
    \item \cite{DDSP_Traditional}はエレキバスから韓国のコムンゴを生成できるシンセサイザーに
    焦点を当てている。
    \item \cite{DDSP_Guitar_Polyphonic}はアコースティックギターのDDSPシンセサイザーで
    ポリフォニーを作るのに焦点を当てている。
\end{itemize}

\section*{考えていること}

DDSPの発表以来、人気が上がってきて、いくつかの興味深いプロジェクトも
毎年生み出されていることを確かめることができた。しかし、録音されたオーディオの処理と
演奏で使用できるシンセサイザーの開発はかなり異なるため、出力の質は特に気になる。

このプロジェクトの目標は弾くのが楽しいシンセサイザー、作曲家やサウンドデザイナーが利用できる
ソフトウェアを開発するであるため、DDSPシンセサイザーはどれぐらい楽器として使用できるかも考慮する必要がある。

\section*{やること}
% Outline future steps here.
\begin{itemize}
    \item DDSPのライブラリーの勉強を続ける
    \item デジタル楽器のデザインや使用についての参考を探す
\end{itemize}

% Add your bibliography here.
\bibliography{../ref}
\bibliographystyle{apalike}

\end{document}